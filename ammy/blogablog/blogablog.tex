%arara: lualatex: { shell: yes }
\documentclass{article}

%%%%% IMPORTS %%%%%%%%%%%%%%%%%%%%%%%%%%%%%%%%%%%%%%%%%%%%%%%%%%%%%%%%%%%%%%%%%%

\usepackage{xparse}
\usepackage[letterpaper, margin=1in]{geometry}
\usepackage{amsmath}
\usepackage{amssymb}
\usepackage[normalem]{ulem}

\usepackage{minted}

%%%%% SETUP AND DEFAULTS %%%%%%%%%%%%%%%%%%%%%%%%%%%%%%%%%%%%%%%%%%%%%%%%%%%%%%%

\setminted{
  linenos
}
\RenewDocumentCommand\thesection{}{Day \arabic{section}:}

%%%%% TITLE AND AUTHOR %%%%%%%%%%%%%%%%%%%%%%%%%%%%%%%%%%%%%%%%%%%%%%%%%%%%%%%%%

\title{Ammy's Notes on AOC (\sout{Alexandria Ocasio-Cortez} Advent of Code) 2023}
\author{Ammy Surname}

%%%%% THE ACTUAL DOCUMENT %%%%%%%%%%%%%%%%%%%%%%%%%%%%%%%%%%%%%%%%%%%%%%%%%%%%%%

\begin{document}
\maketitle

\section{Trebuchet?!}
So this was among the first Elixir code i'd ever written, and i really didn't have a good sense of just...how the language really wants to work, yet. But that's okay! We can do things anyway.

i had no real sense of what the difficulty of the problems would be for this, but this one wasn't really bad.

Part one was fairly easy, though i was still very much in object-orientation land. I just stripped every non-digit with a regex and read the first and last:

\begin{minted}{elixir}
def extract_digits(string) do
  digits = String.replace(string, ~r/\D+/, "")
  first = digits |> String.at(0)
  last = digits |> String.at(-1)
  String.to_integer(first <> last)
end
\end{minted}

Accumulate all of the scores, done.

Part two is a bit trickier; I believe that everyone fell into the same trap, of not realizing that numbers that are smashed together should resolve to two numbers (e.g. "twone" is "21", not "tw1" or "2ne").

I just figured out all of the possible numbers that this can be true for between 1-9 and did the stupid thing:

\begin{minted}{elixir}
def numberize(string) do
  out = String.replace(string, "oneight", "18")
  out = String.replace(out, "twone", "21")
  out = String.replace(out, "threeight", "38")
  out = String.replace(out, "fiveight", "58")
  out = String.replace(out, "sevenine", "79")
  out = String.replace(out, "eightwo", "82")
  out = String.replace(out, "eighthree", "83")
  out = String.replace(out, "nineight", "98")
  out = String.replace(out, "nine", "9")
  out = String.replace(out, "eight", "8")
  out = String.replace(out, "seven", "7")
  out = String.replace(out, "six", "6")
  out = String.replace(out, "five", "5")
  out = String.replace(out, "four", "4")
  out = String.replace(out, "three", "3")
  out = String.replace(out, "two", "2")
  out = String.replace(out, "one", "1")
  out
end
\end{minted}

Some of the other solutions i've seen are more elegant, but this worked for a "learn a new language" scale problem, and now we're on to day two.

\section{Cube Conundrum}

This was the point at which i realized that, at times, part two of a problem would not be "reframe part one"; part two here asks a fundamentally different question using the same data.

But both can reuse the same parsing function, the meat of which is this (after some manipulation of the input file to get it into lines):
\begin{minted}{elixir}
results =
  String.split(l, ";", trim: true)
  |> List.flatten()
  |> Enum.map(fn l -> String.trim(l) end)
  |> Enum.map(fn l -> String.split(l, ",") end)
  |> List.flatten()
  |> Enum.map(fn l -> String.trim(l) end)
  |> Enum.reduce(%{}, fn str, map ->
    [count_str, color] = String.split(str, " ")
    count = String.to_integer(count_str)

    map |> Map.update(color, count, fn val -> max(val, count) end)
  end)
\end{minted}

This feels a lot more "functional programming yay!" than day 1 did, and i'm reasonably happy with it. (i also do not think both calls to `flatten` are necessary...)

After generating a map of colors to their number of occurrences, we can \textit{either} check those occurrences against a set of maxima for part one, or take the product of their minima for part two.

Not bad at all.

\section{Gear Ratios}
TBD
\section{Scratchcards}
TBD
\section{If You Give A Seed A Fertilizer}
TBD
\section{Wait For It}
TBD
\section{Camel Cards}
TBD
\section{Haunted Wasteland}
TBD
\section{Mirage Maintenance}
TBD
\section{Pipe Maze}
TBD



\end{document}