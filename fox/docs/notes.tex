%arara: lualatex: { shell: yes }
\documentclass{article}

%%%%% IMPORTS %%%%%%%%%%%%%%%%%%%%%%%%%%%%%%%%%%%%%%%%%%%%%%%%%%%%%%%%%%%%%%%%%%

\usepackage{xparse}
\usepackage[letterpaper, margin=1in]{geometry}
\usepackage{amsmath}
\usepackage{amssymb}
\usepackage{emoji}
\usepackage{tikz}


\usepackage{minted}

%%%%% SETUP AND DEFAULTS %%%%%%%%%%%%%%%%%%%%%%%%%%%%%%%%%%%%%%%%%%%%%%%%%%%%%%%

\setminted{
  linenos
}

% This overwrites LaTeX's mechanism for writing out section names, which is a
% totally normal and reasonable thing to do. Basically, this makes the section
% counter print "Day 1." instead of just "1."
\RenewDocumentCommand\thesection{}{Day \arabic{section}:}
\RenewDocumentCommand\thesubsection{}{Day \arabic{section}.\arabic{subsection}:}

\usetikzlibrary{datavisualization.formats.functions}
\usetikzlibrary{cd}

%%%%% TITLE AND AUTHOR %%%%%%%%%%%%%%%%%%%%%%%%%%%%%%%%%%%%%%%%%%%%%%%%%%%%%%%%%

\title{Notes on Advent of Code 2023}
\author{Fox Huston}

%%%%% THE ACTUAL DOCUMENT %%%%%%%%%%%%%%%%%%%%%%%%%%%%%%%%%%%%%%%%%%%%%%%%%%%%%%

\begin{document}
\maketitle

\setcounter{section}{5}
\section{Wait For It}

I solved this one with \emoji{sparkles} m a t h e m a t i c s \emoji{sparkles}. I noticed that, if we ignore that the timesteps are discrete integers, this seems a lot like a physics problem. And since the problem is kind enough to add 1 acceleration-unit per input-time-unit, we can write out a formula for the distance the boat travels dependent on: the total time of the race, and how much of that time was spent holding down the charge button:

$$
  \textsf{dist}(c, t) = \int_0^{(t-c)} c dt = c(t-c) = -c^2 + tc,
$$

where $t$ is the total race time, and $c$ is the charge time. For the first
example, this forms a parabola that looks like:

\begin{tikzpicture}[scale=.7]
\datavisualization [school book axes,
                    visualize as smooth line/.list={fn, record},
                    x axis={length=3in, ticks=few},
                    y axis={length=3in, ticks=few, min value=-10, max value=20},
                    style sheet=strong colors,
                    data/format=function ]
data [set=fn] {
  var x : interval [-1:7.5];
  func y = \value x*(7 - \value x);
}
data [set=record] {
  var x : interval [-5:10];
  func y = 9;
};
\end{tikzpicture}

The problem tells us that we need to do better than the record time of 9ms; that is, in chart, we must be (strictly) above the line. So for some record time $r$, this works out to trying to solve

$$
  \textsf{dist}(c, t) > r.
$$

But for now, let's just find the points at which the line $y = r$ intersects $\textsf{dist}(c, t)$: $-c^2 + tc = r$, or rather

$$
  -c^2 + tc - r = 0.
$$

We can use ye olde Pythagorean Theorem to help us, and write:

$$
  d = \frac{t \pm \sqrt{t^2 - 4r}}{2}.
$$

With this, we have $d$ being a pair of numbers that indicate the range of time where we could hold down the button, and beat \emph{or tie} the current record. But for now, we'll switch over to the implementation.

\begin{minted}{elixir}
@spec find_it(integer(), integer()) :: Range.t()
def find_it(race_time, record) do
  nonsense = :math.sqrt(race_time ** 2.0 - 4.0*record) / 2.0

  top = ((race_time/2.0) + nonsense - 0.00001)
  bot = ((race_time/2.0) - nonsense + 0.00001)

  (trunc(bot)+1..trunc(top))
end
\end{minted}

This is the whole thing! It breaks out $d$ into a variable called ``nonsense'' as well as the upper and lower bounds. Now, since we need to always exceed $r$, and since the problem only cares about discrete values between the two points, we can actually just cheat a little bit, and nudge the left-most value a little bit to the right, and the rightmost value a little bit to the left, and rely on the truncation function (converts a float to an int by just lopping off the decimal part) to get us the correct bounds.

And it does!


\setcounter{section}{7}
\section{Haunted Wasteland}
\begin{figure*}[h]
  \centering
  % https://q.uiver.app/#q=WzAsNyxbMSwwLCJcXGRvdHsxfSJdLFsyLDEsIjIiXSxbMSwyLCIzIl0sWzAsMSwiNCJdLFs0LDAsIjEiXSxbNSwxLCIyIl0sWzMsMSwiXFxkb3R7M30iXSxbMCwxXSxbMSwyXSxbMiwzXSxbMywwXSxbNCw1XSxbNSw2XSxbNiw0XV0=
\[\begin{tikzcd}
	& {\dot{1}} &&& 1 \\
	4 && 2 & {\dot{3}} && 2 \\
	& 3
	\arrow[from=1-2, to=2-3]
	\arrow[from=2-3, to=3-2]
	\arrow[from=3-2, to=2-1]
	\arrow[from=2-1, to=1-2]
	\arrow[from=1-5, to=2-6]
	\arrow[from=2-6, to=2-4]
	\arrow[from=2-4, to=1-5]
\end{tikzcd}\]

\caption{Cycles with accept states ($\dot{n}$)}
\end{figure*}


\NewDocumentCommand\SeqCompare{m m}{\genfrac{}{}{0pt}{0}{#1}{#2}}


\paragraph{My Final Answer} This was derived by a lot of trial and error,
 and working with the examples in this document. I would like to prove that this
 does what I say it does, but for now: repeating sequences are represented as
 pairs of $\langle \lambda, \phi \rangle$, where $\lambda$ is the length of
 the sequence and $\phi$ is the accept state (which corresponds really nicely
 to phase in a modular-arithmetic sense, which I describe below. For the
 purposes of making writing things easier, I'll define $(-)_n$ to mean ``the
 expression in parenthesis mod $n$,'' with anything outside the parens not
 included.)

 We want to be able to two concatenate sequences $\langle a, b \rangle \oplus
 \langle c, d \rangle$, so that we get some new sequence such that the
 length of the repeating sequence is the same as the number of steps it would
 take before both sequences restarted simultaneously, and we want the phase
 to be the step at which both sequences would have been in their accept
 states. So, I'll define concatenation as:

 \[
  \langle a, b \rangle \oplus \langle c, d \rangle =
  \langle ab, a \big[ (d \cdot (-1^a)) \% c \big] \rangle.
\]


\paragraph{Observations.} For each loop of $\dot{1}234$, the sequence $12\dot{3}$ ``shifts backwards'' by one---the difference in lengths of the cycles. Is this always true? \bullet{} Given this difference in length (and where the accept states are in each sequence), when will the accept states line up? \bullet{} In this example, it takes 8 steps for the accept states to line up\dots Or rather, it takes 3 times around $12\dot{3}$. I can write this schematically like this:

\[
  \texttt{4032} \qquad
  \SeqCompare{\dot{1}234}{12\dot{3}\textcolor{red}{1}} \qquad
  \SeqCompare{\dot{1}234}{\textcolor{red}{2\dot{3}}12} \qquad
  \SeqCompare{\dot{\mathbf{1}}234}{\dot{\mathbf{3}}\textcolor{red}{12\dot{3}}} \qquad
  \text{(8 steps.)}
\]

(The red text just shows alternate loops of the shorter sequence across the
breaks.) Each of these blocks show one loop around \emph{the longer} sequence. What if we move the dots? Let's say the longer sequence is now $1\dot{2}34$:

\[
  \texttt{4132} \qquad
  \SeqCompare{1\dot{2}34}{12\dot{3}\textcolor{red}{1}} \qquad
  \SeqCompare{1\dot{\mathbf{2}}34}{\textcolor{red}{2\dot{\mathbf{3}}}12} \qquad
  \SeqCompare{1\dot{2}34}{\dot{3}\textcolor{red}{12\dot{3}}} \qquad
  \text{(5 steps.)}
\]


So here's the intuition: each loop through the longer??? one, the distance between the two accept states gets closer. By how much? Is it by $\big| |a| - |b| \big|$, maybe?


\NewDocumentCommand{\Seq}{m O{\lambda} O{\phi}}{#1_{#2#3}}
\NewDocumentCommand{\Seqn}{m m}{\Seq{#1}[\lambda_{#2}][\phi_{#2}]}

The $n^\text{th}$ step of each sequence can be described as $\Seq{s}(n) = (n +
\phi) \mod m$ (and $\phi < \lambda$), where $\lambda$ is the total length of the
sequence, and $\phi$ is the accept state.


% If this is true, then what we want to do, for any two sequences, is find $n$ such that
% \[
%   \Seqn{s}{1}(n) = \Seqn{s}{2}(n).
% \]

Any two sequences $s, t$ form a new sequence

\[
  \Seqn{u}{u} = \Seqn{s}{1} \oplus \Seqn{t}{2},
\]

(assuming $\lambda_1 > \lambda_2$) with its own length \emph{and} phase. The
period (length) of $u$ is always $\lambda_1 \cdot \lambda_2$; if the phases are
equal, then the new phase is just $\phi_u = \phi_1 = \phi_2$. In general,
$\phi_u$ is $\lambda_1 * (\lambda_2 - \phi_2) + \phi_1$???

Notes:
\begin{itemize}
  \item Each sequence is uniquely determined by the pair $\langle \lambda, \phi \rangle$.
  \item Sequences with $n$ accept states are just sets of sequences $\{ \langle \lambda, \phi_1 \rangle, \dots \langle \lambda, \phi_n \rangle \}$.
  \item $\lambda_2$ \emph{cannot evenly divide} $\lambda_1$, since there will be no ``phase shift'' across the sequence. (However, if $\phi_1 = \phi_2$, then a new sequence can still be created; it is just $s$???).
  \item Phase Shift = $\lambda_1 - 2\lambda_2$?. So... the phases unify to this?
  \item I've taken to labeling a comparison $\langle \lambda_1, \phi_1 \rangle, \langle \lambda_2, \phi_2 \rangle$ as the concatenation of these numbers in a typewriter font. For instance, comparing $\langle 5, 0 \rangle$ and $\langle 3, 2 \rangle$, I'll write \texttt{5032}.
\end{itemize}



\[
  \texttt{5032} \qquad
  \SeqCompare{\dot{1}2345}{12\dot{3}\textcolor{red}{12}} \qquad
  \SeqCompare{\dot{1}2345}{\textcolor{red}{\dot{3}}12\dot{3}\textcolor{red}{1}} \qquad
  \SeqCompare{\dot{1}2345}{\textcolor{red}{2\dot{3}}12\dot{3}} \qquad
  \text{(5 steps.)}
\]

% So I think it might be just: if $a$ and $b$ are sequences, and
% $\delta$ is the (shortest) distance between two accept states in two sequences,
% then the total number of steps to get there will be

% \[
%   \frac{\big| |a| - |b| \big| \cdot \max(|a|, |b|)}{\delta} + \dot{a},
% \]
% where $\dot{a}$ is the index of the accept point in the longer sequence.

% Does this work for our examples so far?

\[
  \texttt{5031} \qquad
  \SeqCompare{\dot{1}2345}
    {1\dot{2}3\textcolor{red}{1\dot{2}}} \qquad
  \SeqCompare{\dot{1}2345}
    {\textcolor{red}{3}1\dot{2}3\textcolor{red}{1}} \qquad
  \SeqCompare{\dot{1}2345}
    {\textcolor{red}{\dot{2}3}1\dot{2}3} \qquad
  \text{(10 steps.)}
\]

\NewDocumentCommand\rr{m}{\textcolor{red}{#1}}


\[
\SeqCompare{\dot{1}2345}{\SeqCompare{1\dot{2}34\textcolor{red}{1}}{1\dot{2}3\textcolor{red}{1}\textcolor{red}{\dot{2}}}}
\qquad
\SeqCompare{\textcolor{red}{\dot{1}}\textcolor{red}{2}\textcolor{red}{3}\textcolor{red}{4}\textcolor{red}{5}}{\SeqCompare{\textcolor{red}{\dot{2}}\textcolor{red}{3}\textcolor{red}{4}1\dot{2}}{\textcolor{red}{3}1\dot{2}3\textcolor{red}{1}}}
\qquad
\SeqCompare{\dot{1}2345}{\SeqCompare{34\textcolor{red}{1}\textcolor{red}{\dot{2}}\textcolor{red}{3}}{\textcolor{red}{\dot{2}}\textcolor{red}{3}1\dot{2}3}}
\qquad
\SeqCompare{\textcolor{red}{\dot{1}}\textcolor{red}{2}\textcolor{red}{3}\textcolor{red}{4}\textcolor{red}{5}}{\SeqCompare{\textcolor{red}{4}1\dot{2}34}{\textcolor{red}{1}\textcolor{red}{\dot{2}}\textcolor{red}{3}1\dot{2}}}
\qquad
\SeqCompare{\dot{1}2345}{\SeqCompare{\textcolor{red}{1}\textcolor{red}{\dot{2}}\textcolor{red}{3}\textcolor{red}{4}1}{3\textcolor{red}{1}\textcolor{red}{\dot{2}}\textcolor{red}{3}1}}
\qquad
\SeqCompare{\textcolor{red}{\dot{1}}\textcolor{red}{2}\textcolor{red}{3}\textcolor{red}{4}\textcolor{red}{5}}{\SeqCompare{\dot{2}34\textcolor{red}{1}\textcolor{red}{\dot{2}}}{\dot{2}3\textcolor{red}{1}\textcolor{red}{\dot{2}}\textcolor{red}{3}}}
\qquad
\SeqCompare{\dot{1}2345}{\SeqCompare{\textcolor{red}{3}\textcolor{red}{4}1\dot{2}3}{1\dot{2}3\textcolor{red}{1}\textcolor{red}{\dot{2}}}}
\qquad
\SeqCompare{\textcolor{red}{\dot{1}}\textcolor{red}{2}\textcolor{red}{3}\textcolor{red}{4}\textcolor{red}{5}}{\SeqCompare{4\textcolor{red}{1}\textcolor{red}{\dot{2}}\textcolor{red}{3}\textcolor{red}{4}}{\textcolor{red}{3}1\dot{2}3\textcolor{red}{1}}}
\qquad
\SeqCompare{\dot{1}2345}{\SeqCompare{1\dot{2}34\textcolor{red}{1}}{\textcolor{red}{\dot{2}}\textcolor{red}{3}1\dot{2}3}}
\qquad
\SeqCompare{\textcolor{red}{\dot{1}}\textcolor{red}{2}\textcolor{red}{3}\textcolor{red}{4}\textcolor{red}{5}}{\SeqCompare{\textcolor{red}{\dot{2}}\textcolor{red}{3}\textcolor{red}{4}1\dot{2}}{\textcolor{red}{1}\textcolor{red}{\dot{2}}\textcolor{red}{3}1\dot{2}}}
\]
\[
\SeqCompare{\dot{1}2345}{\SeqCompare{34\textcolor{red}{1}\textcolor{red}{\dot{2}}\textcolor{red}{3}}{3\textcolor{red}{1}\textcolor{red}{\dot{2}}\textcolor{red}{3}1}}
\qquad
\SeqCompare{\textcolor{red}{\dot{1}}\textcolor{red}{2}\textcolor{red}{3}\textcolor{red}{4}\textcolor{red}{5}}{\SeqCompare{\textcolor{red}{4}1\dot{2}34}{\dot{2}3\textcolor{red}{1}\textcolor{red}{\dot{2}}\textcolor{red}{3}}}
\qquad
\SeqCompare{\dot{1}2345}{\SeqCompare{\textcolor{red}{1}\textcolor{red}{\dot{2}}\textcolor{red}{3}\textcolor{red}{4}1}{1\dot{2}3\textcolor{red}{1}\textcolor{red}{\dot{2}}}}
\qquad
\SeqCompare{\textcolor{red}{\dot{1}}\textcolor{red}{2}\textcolor{red}{3}\textcolor{red}{4}\textcolor{red}{5}}{\SeqCompare{\dot{2}34\textcolor{red}{1}\textcolor{red}{\dot{2}}}{\textcolor{red}{3}1\dot{2}3\textcolor{red}{1}}}
\qquad
\SeqCompare{\dot{1}2345}{\SeqCompare{\textcolor{red}{3}\textcolor{red}{4}1\dot{2}3}{\textcolor{red}{\dot{2}}\textcolor{red}{3}1\dot{2}3}}
\qquad
\SeqCompare{\textcolor{red}{\dot{1}}\textcolor{red}{2}\textcolor{red}{3}\textcolor{red}{4}\textcolor{red}{5}}{\SeqCompare{4\textcolor{red}{1}\textcolor{red}{\dot{2}}\textcolor{red}{3}\textcolor{red}{4}}{\textcolor{red}{1}\textcolor{red}{\dot{2}}\textcolor{red}{3}1\dot{2}}}
\qquad
\SeqCompare{\dot{1}2345}{\SeqCompare{1\dot{2}34\textcolor{red}{1}}{3\textcolor{red}{1}\textcolor{red}{\dot{2}}\textcolor{red}{3}1}}
\qquad
\SeqCompare{\textcolor{red}{\dot{1}}\textcolor{red}{2}\textcolor{red}{3}\textcolor{red}{4}\textcolor{red}{5}}{\SeqCompare{\textcolor{red}{\dot{2}}\textcolor{red}{3}\textcolor{red}{4}1\dot{2}}{\dot{2}3\textcolor{red}{1}\textcolor{red}{\dot{2}}\textcolor{red}{3}}}
\qquad
\SeqCompare{\dot{1}2345}{\SeqCompare{34\textcolor{red}{1}\textcolor{red}{\dot{2}}\textcolor{red}{3}}{1\dot{2}3\textcolor{red}{1}\textcolor{red}{\dot{2}}}}
\qquad
\SeqCompare{\textcolor{red}{\dot{1}}\textcolor{red}{2}\textcolor{red}{3}\textcolor{red}{4}\textcolor{red}{5}}{\SeqCompare{\textcolor{red}{4}1\dot{2}34}{\textcolor{red}{3}1\dot{2}3\textcolor{red}{1}}}
\]

% \[ \SeqCompare{\dot{1}23456}{1\dot{2}3\textcolor{red}{1}\textcolor{red}{\dot{2}}\textcolor{red}{3}}
\qquad
\SeqCompare{\textcolor{red}{\dot{1}}\textcolor{red}{2}\textcolor{red}{3}\textcolor{red}{4}\textcolor{red}{5}\textcolor{red}{6}}{1\dot{2}3\textcolor{red}{1}\textcolor{red}{\dot{2}}\textcolor{red}{3}}
\qquad
\SeqCompare{\dot{1}23456}{1\dot{2}3\textcolor{red}{1}\textcolor{red}{\dot{2}}\textcolor{red}{3}}
\qquad
\SeqCompare{\textcolor{red}{\dot{1}}\textcolor{red}{2}\textcolor{red}{3}\textcolor{red}{4}\textcolor{red}{5}\textcolor{red}{6}}{1\dot{2}3\textcolor{red}{1}\textcolor{red}{\dot{2}}\textcolor{red}{3}}
\qquad
\SeqCompare{\dot{1}23456}{1\dot{2}3\textcolor{red}{1}\textcolor{red}{\dot{2}}\textcolor{red}{3}}
\qquad
\SeqCompare{\textcolor{red}{\dot{1}}\textcolor{red}{2}\textcolor{red}{3}\textcolor{red}{4}\textcolor{red}{5}\textcolor{red}{6}}{1\dot{2}3\textcolor{red}{1}\textcolor{red}{\dot{2}}\textcolor{red}{3}}
\qquad
\SeqCompare{\dot{1}23456}{1\dot{2}3\textcolor{red}{1}\textcolor{red}{\dot{2}}\textcolor{red}{3}}
\qquad
\SeqCompare{\textcolor{red}{\dot{1}}\textcolor{red}{2}\textcolor{red}{3}\textcolor{red}{4}\textcolor{red}{5}\textcolor{red}{6}}{1\dot{2}3\textcolor{red}{1}\textcolor{red}{\dot{2}}\textcolor{red}{3}}
\qquad
\SeqCompare{\dot{1}23456}{1\dot{2}3\textcolor{red}{1}\textcolor{red}{\dot{2}}\textcolor{red}{3}}
\qquad
\SeqCompare{\textcolor{red}{\dot{1}}\textcolor{red}{2}\textcolor{red}{3}\textcolor{red}{4}\textcolor{red}{5}\textcolor{red}{6}}{1\dot{2}3\textcolor{red}{1}\textcolor{red}{\dot{2}}\textcolor{red}{3}} \]
\[ \SeqCompare{\dot{1}23456}{1\dot{2}3\textcolor{red}{1}\textcolor{red}{\dot{2}}\textcolor{red}{3}}
\qquad
\SeqCompare{\textcolor{red}{\dot{1}}\textcolor{red}{2}\textcolor{red}{3}\textcolor{red}{4}\textcolor{red}{5}\textcolor{red}{6}}{1\dot{2}3\textcolor{red}{1}\textcolor{red}{\dot{2}}\textcolor{red}{3}}
\qquad
\SeqCompare{\dot{1}23456}{1\dot{2}3\textcolor{red}{1}\textcolor{red}{\dot{2}}\textcolor{red}{3}}
\qquad
\SeqCompare{\textcolor{red}{\dot{1}}\textcolor{red}{2}\textcolor{red}{3}\textcolor{red}{4}\textcolor{red}{5}\textcolor{red}{6}}{1\dot{2}3\textcolor{red}{1}\textcolor{red}{\dot{2}}\textcolor{red}{3}}
\qquad
\SeqCompare{\dot{1}23456}{1\dot{2}3\textcolor{red}{1}\textcolor{red}{\dot{2}}\textcolor{red}{3}}
\qquad
\SeqCompare{\textcolor{red}{\dot{1}}\textcolor{red}{2}\textcolor{red}{3}\textcolor{red}{4}\textcolor{red}{5}\textcolor{red}{6}}{1\dot{2}3\textcolor{red}{1}\textcolor{red}{\dot{2}}\textcolor{red}{3}}
\qquad
\SeqCompare{\dot{1}23456}{1\dot{2}3\textcolor{red}{1}\textcolor{red}{\dot{2}}\textcolor{red}{3}}
\qquad
\SeqCompare{\textcolor{red}{\dot{1}}\textcolor{red}{2}\textcolor{red}{3}\textcolor{red}{4}\textcolor{red}{5}\textcolor{red}{6}}{1\dot{2}3\textcolor{red}{1}\textcolor{red}{\dot{2}}\textcolor{red}{3}}
\qquad
\SeqCompare{\dot{1}23456}{1\dot{2}3\textcolor{red}{1}\textcolor{red}{\dot{2}}\textcolor{red}{3}}
\qquad
\SeqCompare{\textcolor{red}{\dot{1}}\textcolor{red}{2}\textcolor{red}{3}\textcolor{red}{4}\textcolor{red}{5}\textcolor{red}{6}}{1\dot{2}3\textcolor{red}{1}\textcolor{red}{\dot{2}}\textcolor{red}{3}} \]

% \[ \SeqCompare{\dot{1}234567}{1\dot{2}3\textcolor{red}{1}\textcolor{red}{\dot{2}}\textcolor{red}{3}1}
\qquad
\SeqCompare{\textcolor{red}{\dot{1}}\textcolor{red}{2}\textcolor{red}{3}\textcolor{red}{4}\textcolor{red}{5}\textcolor{red}{6}\textcolor{red}{7}}{\dot{2}3\textcolor{red}{1}\textcolor{red}{\dot{2}}\textcolor{red}{3}1\dot{2}}
\qquad
\SeqCompare{\dot{1}234567}{3\textcolor{red}{1}\textcolor{red}{\dot{2}}\textcolor{red}{3}1\dot{2}3}
\qquad
\SeqCompare{\textcolor{red}{\dot{1}}\textcolor{red}{2}\textcolor{red}{3}\textcolor{red}{4}\textcolor{red}{5}\textcolor{red}{6}\textcolor{red}{7}}{\textcolor{red}{1}\textcolor{red}{\dot{2}}\textcolor{red}{3}1\dot{2}3\textcolor{red}{1}}
\qquad
\SeqCompare{\dot{1}234567}{\textcolor{red}{\dot{2}}\textcolor{red}{3}1\dot{2}3\textcolor{red}{1}\textcolor{red}{\dot{2}}}
\qquad
\SeqCompare{\textcolor{red}{\dot{1}}\textcolor{red}{2}\textcolor{red}{3}\textcolor{red}{4}\textcolor{red}{5}\textcolor{red}{6}\textcolor{red}{7}}{\textcolor{red}{3}1\dot{2}3\textcolor{red}{1}\textcolor{red}{\dot{2}}\textcolor{red}{3}}
\qquad
\SeqCompare{\dot{1}234567}{1\dot{2}3\textcolor{red}{1}\textcolor{red}{\dot{2}}\textcolor{red}{3}1}
\qquad
\SeqCompare{\textcolor{red}{\dot{1}}\textcolor{red}{2}\textcolor{red}{3}\textcolor{red}{4}\textcolor{red}{5}\textcolor{red}{6}\textcolor{red}{7}}{\dot{2}3\textcolor{red}{1}\textcolor{red}{\dot{2}}\textcolor{red}{3}1\dot{2}} \]
\[ \SeqCompare{\dot{1}234567}{3\textcolor{red}{1}\textcolor{red}{\dot{2}}\textcolor{red}{3}1\dot{2}3}
\qquad
\SeqCompare{\textcolor{red}{\dot{1}}\textcolor{red}{2}\textcolor{red}{3}\textcolor{red}{4}\textcolor{red}{5}\textcolor{red}{6}\textcolor{red}{7}}{\textcolor{red}{1}\textcolor{red}{\dot{2}}\textcolor{red}{3}1\dot{2}3\textcolor{red}{1}}
\qquad
\SeqCompare{\dot{1}234567}{\textcolor{red}{\dot{2}}\textcolor{red}{3}1\dot{2}3\textcolor{red}{1}\textcolor{red}{\dot{2}}}
\qquad
\SeqCompare{\textcolor{red}{\dot{1}}\textcolor{red}{2}\textcolor{red}{3}\textcolor{red}{4}\textcolor{red}{5}\textcolor{red}{6}\textcolor{red}{7}}{\textcolor{red}{3}1\dot{2}3\textcolor{red}{1}\textcolor{red}{\dot{2}}\textcolor{red}{3}}
\qquad
\SeqCompare{\dot{1}234567}{1\dot{2}3\textcolor{red}{1}\textcolor{red}{\dot{2}}\textcolor{red}{3}1}
\qquad
\SeqCompare{\textcolor{red}{\dot{1}}\textcolor{red}{2}\textcolor{red}{3}\textcolor{red}{4}\textcolor{red}{5}\textcolor{red}{6}\textcolor{red}{7}}{\dot{2}3\textcolor{red}{1}\textcolor{red}{\dot{2}}\textcolor{red}{3}1\dot{2}}
\qquad
\SeqCompare{\dot{1}234567}{3\textcolor{red}{1}\textcolor{red}{\dot{2}}\textcolor{red}{3}1\dot{2}3}
\qquad
\SeqCompare{\textcolor{red}{\dot{1}}\textcolor{red}{2}\textcolor{red}{3}\textcolor{red}{4}\textcolor{red}{5}\textcolor{red}{6}\textcolor{red}{7}}{\textcolor{red}{1}\textcolor{red}{\dot{2}}\textcolor{red}{3}1\dot{2}3\textcolor{red}{1}} \]
\[ \SeqCompare{\dot{1}234567}{\textcolor{red}{\dot{2}}\textcolor{red}{3}1\dot{2}3\textcolor{red}{1}\textcolor{red}{\dot{2}}}
\qquad
\SeqCompare{\textcolor{red}{\dot{1}}\textcolor{red}{2}\textcolor{red}{3}\textcolor{red}{4}\textcolor{red}{5}\textcolor{red}{6}\textcolor{red}{7}}{\textcolor{red}{3}1\dot{2}3\textcolor{red}{1}\textcolor{red}{\dot{2}}\textcolor{red}{3}}
\qquad
\SeqCompare{\dot{1}234567}{1\dot{2}3\textcolor{red}{1}\textcolor{red}{\dot{2}}\textcolor{red}{3}1}
\qquad
\SeqCompare{\textcolor{red}{\dot{1}}\textcolor{red}{2}\textcolor{red}{3}\textcolor{red}{4}\textcolor{red}{5}\textcolor{red}{6}\textcolor{red}{7}}{\dot{2}3\textcolor{red}{1}\textcolor{red}{\dot{2}}\textcolor{red}{3}1\dot{2}} \]

\clearpage

\subsection{Examples}
\begin{figure}[h]
\[ \SeqCompare{\dot{1}234567}{1\dot{2}34\textcolor{red}{1}\textcolor{red}{\dot{2}}\textcolor{red}{3}}
\qquad
\SeqCompare{\textcolor{red}{\dot{1}}\textcolor{red}{2}\textcolor{red}{3}\textcolor{red}{4}\textcolor{red}{5}\textcolor{red}{6}\textcolor{red}{7}}{\textcolor{red}{4}1\dot{2}34\textcolor{red}{1}\textcolor{red}{\dot{2}}}
\qquad
\SeqCompare{\dot{1}234567}{\textcolor{red}{3}\textcolor{red}{4}1\dot{2}34\textcolor{red}{1}}
\qquad
\SeqCompare{\textcolor{red}{\dot{1}}\textcolor{red}{2}\textcolor{red}{3}\textcolor{red}{4}\textcolor{red}{5}\textcolor{red}{6}\textcolor{red}{7}}{\textcolor{red}{\dot{2}}\textcolor{red}{3}\textcolor{red}{4}1\dot{2}34}
\qquad
\SeqCompare{\dot{1}234567}{\textcolor{red}{1}\textcolor{red}{\dot{2}}\textcolor{red}{3}\textcolor{red}{4}1\dot{2}3} \]
\[ \SeqCompare{\textcolor{red}{\dot{1}}\textcolor{red}{2}\textcolor{red}{3}\textcolor{red}{4}\textcolor{red}{5}\textcolor{red}{6}\textcolor{red}{7}}{4\textcolor{red}{1}\textcolor{red}{\dot{2}}\textcolor{red}{3}\textcolor{red}{4}1\dot{2}}
\qquad
\SeqCompare{\dot{1}234567}{34\textcolor{red}{1}\textcolor{red}{\dot{2}}\textcolor{red}{3}\textcolor{red}{4}1}
\qquad
\SeqCompare{\textcolor{red}{\dot{1}}\textcolor{red}{2}\textcolor{red}{3}\textcolor{red}{4}\textcolor{red}{5}\textcolor{red}{6}\textcolor{red}{7}}{\dot{2}34\textcolor{red}{1}\textcolor{red}{\dot{2}}\textcolor{red}{3}\textcolor{red}{4}}
\qquad
\SeqCompare{\dot{1}234567}{1\dot{2}34\textcolor{red}{1}\textcolor{red}{\dot{2}}\textcolor{red}{3}}
\qquad
\SeqCompare{\textcolor{red}{\dot{1}}\textcolor{red}{2}\textcolor{red}{3}\textcolor{red}{4}\textcolor{red}{5}\textcolor{red}{6}\textcolor{red}{7}}{\textcolor{red}{4}1\dot{2}34\textcolor{red}{1}\textcolor{red}{\dot{2}}} \]
\[ \SeqCompare{\dot{1}234567}{\textcolor{red}{3}\textcolor{red}{4}1\dot{2}34\textcolor{red}{1}}
\qquad
\SeqCompare{\textcolor{red}{\dot{1}}\textcolor{red}{2}\textcolor{red}{3}\textcolor{red}{4}\textcolor{red}{5}\textcolor{red}{6}\textcolor{red}{7}}{\textcolor{red}{\dot{2}}\textcolor{red}{3}\textcolor{red}{4}1\dot{2}34}
\qquad
\SeqCompare{\dot{1}234567}{\textcolor{red}{1}\textcolor{red}{\dot{2}}\textcolor{red}{3}\textcolor{red}{4}1\dot{2}3}
\qquad
\SeqCompare{\textcolor{red}{\dot{1}}\textcolor{red}{2}\textcolor{red}{3}\textcolor{red}{4}\textcolor{red}{5}\textcolor{red}{6}\textcolor{red}{7}}{4\textcolor{red}{1}\textcolor{red}{\dot{2}}\textcolor{red}{3}\textcolor{red}{4}1\dot{2}}
\qquad
\SeqCompare{\dot{1}234567}{34\textcolor{red}{1}\textcolor{red}{\dot{2}}\textcolor{red}{3}\textcolor{red}{4}1} \]
\[ \SeqCompare{\textcolor{red}{\dot{1}}\textcolor{red}{2}\textcolor{red}{3}\textcolor{red}{4}\textcolor{red}{5}\textcolor{red}{6}\textcolor{red}{7}}{\dot{2}34\textcolor{red}{1}\textcolor{red}{\dot{2}}\textcolor{red}{3}\textcolor{red}{4}}
\qquad
\SeqCompare{\dot{1}234567}{1\dot{2}34\textcolor{red}{1}\textcolor{red}{\dot{2}}\textcolor{red}{3}}
\qquad
\SeqCompare{\textcolor{red}{\dot{1}}\textcolor{red}{2}\textcolor{red}{3}\textcolor{red}{4}\textcolor{red}{5}\textcolor{red}{6}\textcolor{red}{7}}{\textcolor{red}{4}1\dot{2}34\textcolor{red}{1}\textcolor{red}{\dot{2}}}
\qquad
\SeqCompare{\dot{1}234567}{\textcolor{red}{3}\textcolor{red}{4}1\dot{2}34\textcolor{red}{1}}
\qquad
\SeqCompare{\textcolor{red}{\dot{1}}\textcolor{red}{2}\textcolor{red}{3}\textcolor{red}{4}\textcolor{red}{5}\textcolor{red}{6}\textcolor{red}{7}}{\textcolor{red}{\dot{2}}\textcolor{red}{3}\textcolor{red}{4}1\dot{2}34} \]
\caption{\texttt{7042}}
\end{figure}

\begin{figure}[h]
\[ \SeqCompare{\dot{1}234567}{1\dot{2}3\textcolor{red}{1}\textcolor{red}{\dot{2}}\textcolor{red}{3}1}
\qquad
\SeqCompare{\textcolor{red}{\dot{1}}\textcolor{red}{2}\textcolor{red}{3}\textcolor{red}{4}\textcolor{red}{5}\textcolor{red}{6}\textcolor{red}{7}}{\dot{2}3\textcolor{red}{1}\textcolor{red}{\dot{2}}\textcolor{red}{3}1\dot{2}}
\qquad
\SeqCompare{\dot{1}234567}{3\textcolor{red}{1}\textcolor{red}{\dot{2}}\textcolor{red}{3}1\dot{2}3}
\qquad
\SeqCompare{\textcolor{red}{\dot{1}}\textcolor{red}{2}\textcolor{red}{3}\textcolor{red}{4}\textcolor{red}{5}\textcolor{red}{6}\textcolor{red}{7}}{\textcolor{red}{1}\textcolor{red}{\dot{2}}\textcolor{red}{3}1\dot{2}3\textcolor{red}{1}}
\qquad
\SeqCompare{\dot{1}234567}{\textcolor{red}{\dot{2}}\textcolor{red}{3}1\dot{2}3\textcolor{red}{1}\textcolor{red}{\dot{2}}} \]
\[ \SeqCompare{\textcolor{red}{\dot{1}}\textcolor{red}{2}\textcolor{red}{3}\textcolor{red}{4}\textcolor{red}{5}\textcolor{red}{6}\textcolor{red}{7}}{\textcolor{red}{3}1\dot{2}3\textcolor{red}{1}\textcolor{red}{\dot{2}}\textcolor{red}{3}}
\qquad
\SeqCompare{\dot{1}234567}{1\dot{2}3\textcolor{red}{1}\textcolor{red}{\dot{2}}\textcolor{red}{3}1}
\qquad
\SeqCompare{\textcolor{red}{\dot{1}}\textcolor{red}{2}\textcolor{red}{3}\textcolor{red}{4}\textcolor{red}{5}\textcolor{red}{6}\textcolor{red}{7}}{\dot{2}3\textcolor{red}{1}\textcolor{red}{\dot{2}}\textcolor{red}{3}1\dot{2}}
\qquad
\SeqCompare{\dot{1}234567}{3\textcolor{red}{1}\textcolor{red}{\dot{2}}\textcolor{red}{3}1\dot{2}3}
\qquad
\SeqCompare{\textcolor{red}{\dot{1}}\textcolor{red}{2}\textcolor{red}{3}\textcolor{red}{4}\textcolor{red}{5}\textcolor{red}{6}\textcolor{red}{7}}{\textcolor{red}{1}\textcolor{red}{\dot{2}}\textcolor{red}{3}1\dot{2}3\textcolor{red}{1}} \]
\[ \SeqCompare{\dot{1}234567}{\textcolor{red}{\dot{2}}\textcolor{red}{3}1\dot{2}3\textcolor{red}{1}\textcolor{red}{\dot{2}}}
\qquad
\SeqCompare{\textcolor{red}{\dot{1}}\textcolor{red}{2}\textcolor{red}{3}\textcolor{red}{4}\textcolor{red}{5}\textcolor{red}{6}\textcolor{red}{7}}{\textcolor{red}{3}1\dot{2}3\textcolor{red}{1}\textcolor{red}{\dot{2}}\textcolor{red}{3}}
\qquad
\SeqCompare{\dot{1}234567}{1\dot{2}3\textcolor{red}{1}\textcolor{red}{\dot{2}}\textcolor{red}{3}1}
\qquad
\SeqCompare{\textcolor{red}{\dot{1}}\textcolor{red}{2}\textcolor{red}{3}\textcolor{red}{4}\textcolor{red}{5}\textcolor{red}{6}\textcolor{red}{7}}{\dot{2}3\textcolor{red}{1}\textcolor{red}{\dot{2}}\textcolor{red}{3}1\dot{2}}
\qquad
\SeqCompare{\dot{1}234567}{3\textcolor{red}{1}\textcolor{red}{\dot{2}}\textcolor{red}{3}1\dot{2}3} \]
\[ \SeqCompare{\textcolor{red}{\dot{1}}\textcolor{red}{2}\textcolor{red}{3}\textcolor{red}{4}\textcolor{red}{5}\textcolor{red}{6}\textcolor{red}{7}}{\textcolor{red}{1}\textcolor{red}{\dot{2}}\textcolor{red}{3}1\dot{2}3\textcolor{red}{1}}
\qquad
\SeqCompare{\dot{1}234567}{\textcolor{red}{\dot{2}}\textcolor{red}{3}1\dot{2}3\textcolor{red}{1}\textcolor{red}{\dot{2}}}
\qquad
\SeqCompare{\textcolor{red}{\dot{1}}\textcolor{red}{2}\textcolor{red}{3}\textcolor{red}{4}\textcolor{red}{5}\textcolor{red}{6}\textcolor{red}{7}}{\textcolor{red}{3}1\dot{2}3\textcolor{red}{1}\textcolor{red}{\dot{2}}\textcolor{red}{3}}
\qquad
\SeqCompare{\dot{1}234567}{1\dot{2}3\textcolor{red}{1}\textcolor{red}{\dot{2}}\textcolor{red}{3}1}
\qquad
\SeqCompare{\textcolor{red}{\dot{1}}\textcolor{red}{2}\textcolor{red}{3}\textcolor{red}{4}\textcolor{red}{5}\textcolor{red}{6}\textcolor{red}{7}}{\dot{2}3\textcolor{red}{1}\textcolor{red}{\dot{2}}\textcolor{red}{3}1\dot{2}} \]
\caption{\texttt{7032}}
\end{figure}

\end{document}