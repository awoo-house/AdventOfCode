%arara: lualatex: { shell: yes }
\documentclass{article}

%%%%% IMPORTS %%%%%%%%%%%%%%%%%%%%%%%%%%%%%%%%%%%%%%%%%%%%%%%%%%%%%%%%%%%%%%%%%%

\usepackage{xparse}
\usepackage[letterpaper, margin=1in]{geometry}
\usepackage{amsmath}
\usepackage{amssymb}
\usepackage{emoji}
\usepackage{tikz}

\usepackage{minted}

%%%%% SETUP AND DEFAULTS %%%%%%%%%%%%%%%%%%%%%%%%%%%%%%%%%%%%%%%%%%%%%%%%%%%%%%%

\setminted{
  linenos
}

% This overwrites LaTeX's mechanism for writing out section names, which is a
% totally normal and reasonable thing to do. Basically, this makes the section
% counter print "Day 1." instead of just "1."
\RenewDocumentCommand\thesection{}{Day \arabic{section}:}

\usetikzlibrary {datavisualization.formats.functions}

%%%%% TITLE AND AUTHOR %%%%%%%%%%%%%%%%%%%%%%%%%%%%%%%%%%%%%%%%%%%%%%%%%%%%%%%%%

\title{Notes on Advent of Code 2023}
\author{Fox Huston}

%%%%% THE ACTUAL DOCUMENT %%%%%%%%%%%%%%%%%%%%%%%%%%%%%%%%%%%%%%%%%%%%%%%%%%%%%%

\begin{document}
\maketitle

\setcounter{section}{5}
\section{Wait For It}

I solved this one with \emoji{sparkles} m a t h e m a t i c s \emoji{sparkles}. I noticed that, if we ignore that the timesteps are discrete integers, this seems a lot like a physics problem. And since the problem is kind enough to add 1 acceleration-unit per input-time-unit, we can write out a formula for the distance the boat travels dependent on: the total time of the race, and how much of that time was spent holding down the charge button:

$$
  \textsf{dist}(c, t) = \int_0^{(t-c)} c dt = c(t-c) = -c^2 + tc,
$$

where $t$ is the total race time, and $c$ is the charge time. For the first
example, this forms a parabola that looks like:

\begin{tikzpicture}[scale=.7]
\datavisualization [school book axes,
                    visualize as smooth line/.list={fn, record},
                    x axis={length=3in, ticks=few},
                    y axis={length=3in, ticks=few, min value=-10, max value=20},
                    style sheet=strong colors,
                    data/format=function ]
data [set=fn] {
  var x : interval [-1:7.5];
  func y = \value x*(7 - \value x);
}
data [set=record] {
  var x : interval [-5:10];
  func y = 9;
};
\end{tikzpicture}

The problem tells us that we need to do better than the record time of 9ms; that is, in chart, we must be (strictly) above the line. So for some record time $r$, this works out to trying to solve

$$
  \textsf{dist}(c, t) > r.
$$

But for now, let's just find the points at which the line $y = r$ intersects $\textsf{dist}(c, t)$: $-c^2 + tc = r$, or rather

$$
  -c^2 + tc - r = 0.
$$

We can use ye olde Pythagorean Theorem to help us, and write:

$$
  d = \frac{t \pm \sqrt{t^2 - 4r}}{2}.
$$

With this, we have $d$ being a pair of numbers that indicate the range of time where we could hold down the button, and beat \emph{or tie} the current record. But for now, we'll switch over to the implementation.

\begin{minted}{elixir}
@spec find_it(integer(), integer()) :: Range.t()
def find_it(race_time, record) do
  nonsense = :math.sqrt(race_time ** 2.0 - 4.0*record) / 2.0

  top = ((race_time/2.0) + nonsense - 0.00001)
  bot = ((race_time/2.0) - nonsense + 0.00001)

  (trunc(bot)+1..trunc(top))
end
\end{minted}

This is the whole thing! It breaks out $d$ into a variable called ``nonsense'' as well as the upper and lower bounds. Now, since we need to always exceed $r$, and since the problem only cares about discrete values between the two points, we can actually just cheat a little bit, and nudge the left-most value a little bit to the right, and the rightmost value a little bit to the left, and rely on the truncation function (converts a float to an int by just lopping off the decimal part) to get us the correct bounds.

And it does!

\end{document}