%arara: lualatex: { shell: yes }
\documentclass{article}

%%%%% IMPORTS %%%%%%%%%%%%%%%%%%%%%%%%%%%%%%%%%%%%%%%%%%%%%%%%%%%%%%%%%%%%%%%%%%

\usepackage{xparse}
\usepackage[letterpaper, margin=1in]{geometry}
\usepackage{amsmath}
\usepackage{amssymb}
\usepackage{emoji}
\usepackage{tikz}


\usepackage{minted}

%%%%% SETUP AND DEFAULTS %%%%%%%%%%%%%%%%%%%%%%%%%%%%%%%%%%%%%%%%%%%%%%%%%%%%%%%

\setminted{
  linenos
}

% This overwrites LaTeX's mechanism for writing out section names, which is a
% totally normal and reasonable thing to do. Basically, this makes the section
% counter print "Day 1." instead of just "1."
\RenewDocumentCommand\thesection{}{Day \arabic{section}:}

\usetikzlibrary{datavisualization.formats.functions}
\usetikzlibrary{cd}

%%%%% TITLE AND AUTHOR %%%%%%%%%%%%%%%%%%%%%%%%%%%%%%%%%%%%%%%%%%%%%%%%%%%%%%%%%

\title{Notes on Advent of Code 2023}
\author{Fox Huston}

%%%%% THE ACTUAL DOCUMENT %%%%%%%%%%%%%%%%%%%%%%%%%%%%%%%%%%%%%%%%%%%%%%%%%%%%%%

\begin{document}
\maketitle

\setcounter{section}{5}
\section{Wait For It}

I solved this one with \emoji{sparkles} m a t h e m a t i c s \emoji{sparkles}. I noticed that, if we ignore that the timesteps are discrete integers, this seems a lot like a physics problem. And since the problem is kind enough to add 1 acceleration-unit per input-time-unit, we can write out a formula for the distance the boat travels dependent on: the total time of the race, and how much of that time was spent holding down the charge button:

$$
  \textsf{dist}(c, t) = \int_0^{(t-c)} c dt = c(t-c) = -c^2 + tc,
$$

where $t$ is the total race time, and $c$ is the charge time. For the first
example, this forms a parabola that looks like:

\begin{tikzpicture}[scale=.7]
\datavisualization [school book axes,
                    visualize as smooth line/.list={fn, record},
                    x axis={length=3in, ticks=few},
                    y axis={length=3in, ticks=few, min value=-10, max value=20},
                    style sheet=strong colors,
                    data/format=function ]
data [set=fn] {
  var x : interval [-1:7.5];
  func y = \value x*(7 - \value x);
}
data [set=record] {
  var x : interval [-5:10];
  func y = 9;
};
\end{tikzpicture}

The problem tells us that we need to do better than the record time of 9ms; that is, in chart, we must be (strictly) above the line. So for some record time $r$, this works out to trying to solve

$$
  \textsf{dist}(c, t) > r.
$$

But for now, let's just find the points at which the line $y = r$ intersects $\textsf{dist}(c, t)$: $-c^2 + tc = r$, or rather

$$
  -c^2 + tc - r = 0.
$$

We can use ye olde Pythagorean Theorem to help us, and write:

$$
  d = \frac{t \pm \sqrt{t^2 - 4r}}{2}.
$$

With this, we have $d$ being a pair of numbers that indicate the range of time where we could hold down the button, and beat \emph{or tie} the current record. But for now, we'll switch over to the implementation.

\begin{minted}{elixir}
@spec find_it(integer(), integer()) :: Range.t()
def find_it(race_time, record) do
  nonsense = :math.sqrt(race_time ** 2.0 - 4.0*record) / 2.0

  top = ((race_time/2.0) + nonsense - 0.00001)
  bot = ((race_time/2.0) - nonsense + 0.00001)

  (trunc(bot)+1..trunc(top))
end
\end{minted}

This is the whole thing! It breaks out $d$ into a variable called ``nonsense'' as well as the upper and lower bounds. Now, since we need to always exceed $r$, and since the problem only cares about discrete values between the two points, we can actually just cheat a little bit, and nudge the left-most value a little bit to the right, and the rightmost value a little bit to the left, and rely on the truncation function (converts a float to an int by just lopping off the decimal part) to get us the correct bounds.

And it does!


\setcounter{section}{7}
\section{Haunted Wasteland}
\begin{figure*}[h]
  \centering
  % https://q.uiver.app/#q=WzAsNyxbMSwwLCJcXGRvdHsxfSJdLFsyLDEsIjIiXSxbMSwyLCIzIl0sWzAsMSwiNCJdLFs0LDAsIjEiXSxbNSwxLCIyIl0sWzMsMSwiXFxkb3R7M30iXSxbMCwxXSxbMSwyXSxbMiwzXSxbMywwXSxbNCw1XSxbNSw2XSxbNiw0XV0=
\[\begin{tikzcd}
	& {\dot{1}} &&& 1 \\
	4 && 2 & {\dot{3}} && 2 \\
	& 3
	\arrow[from=1-2, to=2-3]
	\arrow[from=2-3, to=3-2]
	\arrow[from=3-2, to=2-1]
	\arrow[from=2-1, to=1-2]
	\arrow[from=1-5, to=2-6]
	\arrow[from=2-6, to=2-4]
	\arrow[from=2-4, to=1-5]
\end{tikzcd}\]

\caption{Cycles with accept states ($\dot{n}$)}
\end{figure*}


\NewDocumentCommand\SeqCompare{m m}{\genfrac{}{}{0pt}{0}{#1}{#2}}

\paragraph{Observations.} For each loop of $\dot{1}234$, the sequence $12\dot{3}$ ``shifts backwards'' by one---the difference in lengths of the cycles. Is this always true? \bullet{} Given this difference in length (and where the accept states are in each sequence), when will the accept states line up? \bullet{} In this example, it takes 8 steps for the accept states to line up\dots Or rather, it takes 3 times around $12\dot{3}$. I can write this schematically like this:

\[
  \SeqCompare{\dot{1}234}{12\dot{3}\textcolor{red}{1}} \qquad
  \SeqCompare{\dot{1}234}{\textcolor{red}{2\dot{3}}12} \qquad
  \SeqCompare{\dot{\mathbf{1}}234}{\dot{\mathbf{3}}\textcolor{red}{12\dot{3}}} \qquad
  \text{(8 steps.)}
\]
(The red text just shows alternate loops of the shorter sequence across the
breaks.) Each of these blocks show one loop around \emph{the longer} sequence. What if we move the dots? Let's say the longer sequence is now $1\dot{2}34$:

\[
  \SeqCompare{1\dot{2}34}{12\dot{3}\textcolor{red}{1}} \qquad
  \SeqCompare{1\dot{\mathbf{2}}34}{\textcolor{red}{2\dot{\mathbf{3}}}12} \qquad
  \SeqCompare{1\dot{2}34}{\dot{3}\textcolor{red}{12\dot{3}}} \qquad
  \text{(5 steps.)}
\]


So here's the intuition: each loop through the longer??? one, the distance between the two accept states gets closer. By how much? Is it by $\big| |a| - |b| \big|$, maybe?


\[
  \SeqCompare{\dot{1}2345}{12\dot{3}\textcolor{red}{12}} \qquad
  \SeqCompare{\dot{1}2345}{\textcolor{red}{\dot{3}}12\dot{3}\textcolor{red}{1}} \qquad
  \SeqCompare{\dot{1}2345}{\textcolor{red}{2\dot{3}}12\dot{3}} \qquad
  \text{(? steps.)}
\]

So I think it might be just: if $a$ and $b$ are sequences, and
$\delta$ is the (shortest) distance between two accept states in two sequences,
then the total number of steps to get there will be

\[
  \frac{\big| |a| - |b| \big| \cdot \max(|a|, |b|)}{\delta} + \dot{a},
\]
where $\dot{a}$ is the index of the accept point in the longer sequence.

Does this work for our examples so far?

\[
  \SeqCompare{\dot{1}2345}
    {1\dot{2}3\textcolor{red}{1\dot{2}}} \qquad
  \SeqCompare{\dot{1}2345}
    {\textcolor{red}{3}1\dot{2}3\textcolor{red}{1}} \qquad
  \SeqCompare{\dot{1}2345}
    {\textcolor{red}{\dot{2}3}1\dot{2}3} \qquad
  \text{(10 steps.)}
\]

\NewDocumentCommand\rr{m}{\textcolor{red}{#1}}

\[
\SeqCompare{\dot{1}2345}{\SeqCompare{1\dot{2}34\textcolor{red}{1}}{1\dot{2}3\textcolor{red}{1}\textcolor{red}{\dot{2}}}}
\qquad
\SeqCompare{\textcolor{red}{\dot{1}}\textcolor{red}{2}\textcolor{red}{3}\textcolor{red}{4}\textcolor{red}{5}}{\SeqCompare{\textcolor{red}{\dot{2}}\textcolor{red}{3}\textcolor{red}{4}1\dot{2}}{\textcolor{red}{3}1\dot{2}3\textcolor{red}{1}}}
\qquad
\SeqCompare{\dot{1}2345}{\SeqCompare{34\textcolor{red}{1}\textcolor{red}{\dot{2}}\textcolor{red}{3}}{\textcolor{red}{\dot{2}}\textcolor{red}{3}1\dot{2}3}}
\qquad
\SeqCompare{\textcolor{red}{\dot{1}}\textcolor{red}{2}\textcolor{red}{3}\textcolor{red}{4}\textcolor{red}{5}}{\SeqCompare{\textcolor{red}{4}1\dot{2}34}{\textcolor{red}{1}\textcolor{red}{\dot{2}}\textcolor{red}{3}1\dot{2}}}
\qquad
\SeqCompare{\dot{1}2345}{\SeqCompare{\textcolor{red}{1}\textcolor{red}{\dot{2}}\textcolor{red}{3}\textcolor{red}{4}1}{3\textcolor{red}{1}\textcolor{red}{\dot{2}}\textcolor{red}{3}1}}
\qquad
\SeqCompare{\textcolor{red}{\dot{1}}\textcolor{red}{2}\textcolor{red}{3}\textcolor{red}{4}\textcolor{red}{5}}{\SeqCompare{\dot{2}34\textcolor{red}{1}\textcolor{red}{\dot{2}}}{\dot{2}3\textcolor{red}{1}\textcolor{red}{\dot{2}}\textcolor{red}{3}}}
\qquad
\SeqCompare{\dot{1}2345}{\SeqCompare{\textcolor{red}{3}\textcolor{red}{4}1\dot{2}3}{1\dot{2}3\textcolor{red}{1}\textcolor{red}{\dot{2}}}}
\qquad
\SeqCompare{\textcolor{red}{\dot{1}}\textcolor{red}{2}\textcolor{red}{3}\textcolor{red}{4}\textcolor{red}{5}}{\SeqCompare{4\textcolor{red}{1}\textcolor{red}{\dot{2}}\textcolor{red}{3}\textcolor{red}{4}}{\textcolor{red}{3}1\dot{2}3\textcolor{red}{1}}}
\qquad
\SeqCompare{\dot{1}2345}{\SeqCompare{1\dot{2}34\textcolor{red}{1}}{\textcolor{red}{\dot{2}}\textcolor{red}{3}1\dot{2}3}}
\qquad
\SeqCompare{\textcolor{red}{\dot{1}}\textcolor{red}{2}\textcolor{red}{3}\textcolor{red}{4}\textcolor{red}{5}}{\SeqCompare{\textcolor{red}{\dot{2}}\textcolor{red}{3}\textcolor{red}{4}1\dot{2}}{\textcolor{red}{1}\textcolor{red}{\dot{2}}\textcolor{red}{3}1\dot{2}}}
\]


\end{document}